\documentclass{article}
\usepackage{graphicx} % Required for inserting images
\usepackage{amsmath,amsthm, amsfonts}
\newtheorem{theorem}{Theorem}[section]
% \newtheorem{what_to_write in code}{what to display}[numbering according to what]
\newtheorem{corollary}{Corollary}[theorem]
\newcommand{\R}{\mathbb{R}}
\newcommand{\cv}[2]{\begin{bmatrix}
#1 \\
#2 \\
\end{bmatrix}}

\title{LATEX TUTORIAL}
\author{neerajnjps03 }
\date{May 2023}

\begin{document}

\maketitle

\section{Introduction}
% FOR un-numbered sections, use \section*{text}
Let's begin with a formula $e^{i\pi} + 1 = 0$

$$e = \lim_{n\to\infty}\left(1 + \frac{1}{n}\right)^n 
 = \lim_{n\to\infty}\frac{n}{\sqrt[n]{n!}}$$
\begin{enumerate}
    \item 
	We can do another:

 $$
    e = \sum_{n=0}^{\infty} \frac{1}{n!}
 $$
\end{enumerate}
\begin{itemize}
    \item \underline{\textit{\textbf{As a continued fraction}}}

    
We can also use continued fraction
$$
    e = 2 + \frac{1}{2 + \frac{2}{3 + \frac{3}{4 + \frac{4}{5 + \ddots}}}}
$$
\end{itemize}

\section*{More formulas}

$$
    \int_a^bf(x)dx
$$
\[
    \int_a^bf(x)dx
\]


\begin{equation}
  \iiint f(x, y, z) dxdydz
  % begin equation allows for referncing
  \label{triple_integral}
\end{equation}


\begin{equation}
\label{eq:vector}
    \Vec{v} = <v_1, v_2, v_3 >
\end{equation}

$$
    \Vec{v}\cdot\Vec{w}
$$
$$
    \begin{bmatrix}
        1 & 2 & 3 \\
        4 & 5 & 6 \\
    \end{bmatrix}
$$

% including images....
Equation \ref{eq:vector} is a vector
\begin{equation}
e = \lim_{n\to\infty}\left(1 + \frac{1}{n}\right)^n 
 = \lim_{n\to\infty}\frac{n}{\sqrt[n]{n!}}
 =\lim_{t\to{0^+}}(1+t)^{\frac{1}{t}}
\end{equation}
\begin{align}
% to indicate line has ended, we have to use \\ 
% to make it even we have to use &
e & = \lim_{n\to\infty}\left(1 + \frac{1}{n}\right)^n \\
 &= \lim_{n\to\infty}\frac{n}{\sqrt[n]{n!}} \\
 &=\lim_{t\to{0^+}}(1+t)^{\frac{1}{t}}
\end{align}


\begin{equation}
\begin{split}
% to indicate line has ended, we have to use \\ 
% to make it even we have to use &
% this begin equation together with begin split, does the same job as begin align but the whole equation is referenced by a single number rather than an individual number for each line
e & = \lim_{n\to\infty}\left(1 + \frac{1}{n}\right)^n \\
 &= \lim_{n\to\infty}\frac{n}{\sqrt[n]{n!}} \\
 &=\lim_{t\to{0^+}}(1+t)^{\frac{1}{t}}
\end{split}
\end{equation}
% \begin{multline}
%     e^x \approx 1 + x + \frac{x^2}{2!} + \frac{x^3}{3!} + \frac{x^4}{4!} + \frac{x^5}{5!} + \frac{x^6}{6!} + \frac{x^7}{7!} + \frac{x^8}{8!} + \frac{x^9}{9!} + \frac{x^{10}}{10!} + \frac{x^{11}}{11!} + \frac{x^{12}}{12!} + \frac{x^{13}}{13!} + \frac{x^{14}}{14!}+ \frac{x^{15}} {15!} + \frac{x^{16}}{16!} + \frac{x^{17}}{17!}+ \frac{x^{18}}{18!}
% \end{multline}
% multline not working ???


\section{More Tricks}

\begin{table}[h]
    \centering
    \begin{tabular}{|c|r|}
    \hline
       1  & 1 \\ \hline
       3  & 4000000000000000 \\
    \hline
    \end{tabular}
    \caption{Caption}
    \label{tab:my_label}
\end{table}

% include graphicx...


% \section{Theorems}
\begin{theorem}[Hello World Theorem] Hello world 
    
\end{theorem}
\begin{proof}
    Proof goes here
    % last symbol will automatically will be the Q.E.D symbol
\end{proof}


\begin{corollary}
    corollary from the theorem
\end{corollary}

% making newcommands

The real numbers $\mathbb{R}$


Instead of typing $\mathbb{R}$ always, we can use macros


The real numbers $\R$

for shorthand for column matrix consisting of 2 elements
Now the shorthand usage

$$
\cv{x}{y}
$$

\end{document}
